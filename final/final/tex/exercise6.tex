
\begin{question}
A treatment group of size 31 has a mean of 9.94 and standard deviation
of 0.647. A control group of size 19 has a mean of 10.3 and standard
deviation of 0.477. If you decided to use a signficance level of 0.02,
is there sufficient evidence to conclude the treatment causes an effect?

By using the Welch-Satterthwaite equation, I've calculated the degrees
of freedom should be 46.
\begin{answerlist}
  \item State the null hypothesis.
  \item State the alternative hypothesis.
  \item Evaluate the critical value. (The critical value is either \(z^\star\)
or \(t^\star\). Determine its value.)
  \item Determine the standard error of the relevant sampling distribution.
  \item Evaluate the absolute value of the test statistic. (The test statistic
is either \(z_\text{obs}\) or \(t_\text{obs}\). Determine its absolute
value.)
  \item If possible, evaluate the \(p\)-value. Otherwise, describe an interval
containing the \(p\)-value.
  \item Do we reject or retain the null?
\end{answerlist}
\end{question}

\begin{solution}
We are given unpaired data. We are considering a difference of means.
Label the given information. \[
\begin{aligned}
n_1 &= 31 \\
\bar{x}_1 &= 9.94\\
s_1 &= 0.647\\
n_2 &= 19 \\
\bar{x}_2 &= 10.3\\
s_2 &= 0.477\\
\alpha &= 0.02 \\
df &= 46
\end{aligned}
\] State the hypotheses. \[H_0:~~\mu_2-\mu_1 = 0\]
\[H_A:~~\mu_2-\mu_1 \ne 0\] We are using a two-tail test. Find
\(t^{\star}\) such that \(P(|T|>t^{\star})=0.02\) by using a \(t\)
table. \[t^{\star} = 2.41\] Calculate the standard error. \[
\begin{aligned}
SE &= \sqrt{\frac{(s_1)^2}{n_1}+\frac{(s_2)^2}{n_2}} \\
&= \sqrt{\frac{(0.647)^2}{31}+\frac{(0.477)^2}{19}} \\
&= 0.16
\end{aligned}
\] Determine the test statistic. \[
\begin{aligned}
t_\text{obs} &= \frac{(\bar{x}_2-\bar{x}_1)-(\mu_2-\mu_1)_0}{SE} \\
&= \frac{(10.3-9.94)-(0)}{0.16} \\
&= 2.26
\end{aligned}
\] Compare \(|t_\text{obs}|\) and \(t^\star\).
\[|t_\text{obs}| < t^\star \] We can determine an interval for the
\(p\)-value using the \(t\) table. \[0.02 < p\text{-value} < 0.04\]
Compare \(p\)-value and \(\alpha\). \[p\text{-value} > \alpha \] We
conclude that we should retain the null hypothesis.
\begin{answerlist}
  \item \(H_0: ~~\mu_2-\mu_1=0\)
  \item \(H_\text{A}:~~ \mu_2-\mu_1\ne 0\)
  \item \(t^\star = 2.41\)
  \item \(SE = 0.16\)
  \item \(|t_\text{obs}| = 2.26\)
  \item \(0.02 < p\text{-value} < 0.04\)
  \item retain the null
\end{answerlist}
\end{solution}

