
\begin{question}
As an ornithologist, you wish to determine the average body mass of
\emph{Piranga rubra}. You randomly sample 31 adults of \emph{Piranga
rubra}, resulting in a sample mean of 37.46 grams and a sample standard
deviation of 6.73 grams. Determine a 99.5\% confidence interval of the
true population mean.
\end{question}

\begin{solution}
We are given the sample size, sample mean, sample standard deviation,
and confidence level. \[
\begin{aligned}
n &= 31 \\
\bar{x} &= 37.46 \\
s &= 6.73\\
CL &= 0.995
\end{aligned}
\] Determine the degrees of freedom (because we don't know \(\sigma\)
and we are doing inference so we need to use the \(t\) distribution).
\[df = n-1 = 30\] Determine the critical \(t\) value, \(t^\star\), such
that \(P(|T|<t^\star) = 0.995\). \[t^\star = 3.03 \] Calculate the
standard error.
\[ SE = \frac{s}{\sqrt{n}} = \frac{6.73}{\sqrt{31}} = 1.21 \] We want to
make an inference about the population mean.
\[\mu ~\approx~ \bar{x}\pm t^\star SE \] Determine the bounds. \[
\begin{aligned}
CI &= (\bar{x} - t^\star SE,\,\bar{x} + t^\star SE) \\
   &= (37.46 - 3.03\times1.21,\,37.46 + 3.03\times1.21) \\
   &= (33.8,\,41.1) 
\end{aligned}
\] We are 99.5\% confident that the population mean is between 33.8 and
41.1.
\end{solution}

