
\begin{question}
An experiment is run with a treatment group of size 168 and a control
group of size 194. The results are summarized in the table below.

\begin{longtable}[]{@{}ccc@{}}
\toprule
& treatment & control\tabularnewline
\midrule
\endhead
special & 151 & 188\tabularnewline
not special & 17 & 6\tabularnewline
\bottomrule
\end{longtable}

Using a significance level of 0.01, determine whether the treatment
causes an effect on the proportion of cases that are special.
\begin{answerlist}
  \item State the null hypothesis.
  \item State the alternative hypothesis.
  \item Evaluate the critical value. (The critical value is either \(z^\star\)
or \(t^\star\). Determine its value.)
  \item Determine the standard error of the relevant sampling distribution.
  \item Evaluate the absolute value of the test statistic. (The test statistic
is either \(z_\text{obs}\) or \(t_\text{obs}\). Determine its absolute
value.)
  \item If possible, evaluate the \(p\)-value. Otherwise, describe an interval
containing the \(p\)-value.
  \item Do we reject or retain the null?
\end{answerlist}
\end{question}

\begin{solution}
State the hypotheses. \[H_0:~~ p_2-p_1 = 0 \]
\[H_\text{A}: ~~ p_2-p_1 \ne 0 \] Find \(z^\star\) such that
\(P(|Z|>z^\star) = 0.01\).
\[z^\star = \Phi^{-1}\left(1-\frac{\alpha}{2}\right) = 2.58 \] Determine
the sample proportions. \[\hat{p}_1 = \frac{151}{168} = 0.899 \]
\[\hat{p}_2 = \frac{188}{194} = 0.969 \] Determine the difference of
sample proportions. \[\hat{p}_2-\hat{p}_1 = 0.969-0.899 = 0.07 \]
Determine the pooled proportion (because the null assumes the population
proportions are equal). \[ \hat{p} = \frac{151+188}{168+194} = 0.936 \]
Determine the standard error. \[
\begin{aligned}
SE &= \sqrt{\frac{\hat{p}(1-\hat{p})}{n_1}+\frac{\hat{p}(1-\hat{p})}{n_2}} \\[1em]
&= \sqrt{\frac{(0.936)(0.064)}{168}+\frac{(0.936)(0.064)}{194}} \\[1em]
&= 0.0258
\end{aligned}
\] We can be more specific about what the null hypothesis claims.
\[H_0:~~~~ \hat{P}_2-\hat{P}_1 ~\sim~ \mathcal{N}(0,\,0.0258)\] We want
to describe how unusual our observation is under the null by finding the
\(p\)-value. To do so, first find the \(z\) score. \[
\begin{aligned}
z &= \frac{(\hat{p}_2-\hat{p}_1)-(p_2-p_1)_0}{SE}\\
&= \frac{(0.969-0.899)-0}{0.0258} \\
&= 2.71
\end{aligned}
\] Determine the \(p\)-value. \[
\begin{aligned}
p\text{-value} &= 2\cdot \Phi(-|z|)\\
&= 2\cdot \Phi(-2.71)\\
&= 0.0068
\end{aligned}
\] Compare the \(p\)-value to the signficance level.
\[ p\text{-value} < \alpha \] So, we reject the null hypothesis. Thus
the difference in proportions is significant.
\begin{answerlist}
  \item \(H_0: ~~p_2-p_1=0\)
  \item \(H_\text{A}:~~ p_2-p_1\ne 0\)
  \item \(z^\star = 2.58\)
  \item \(SE = 0.0258\)
  \item \(|z_\text{obs}| = 2.71\)
  \item \(p\text{-value} = 0.0068\)
  \item reject the null
\end{answerlist}
\end{solution}

