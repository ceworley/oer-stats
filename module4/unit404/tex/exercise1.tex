
\begin{question}
An experiment has \(n_1 = 4\) plants in the treatment group and
\(n_2 = 6\) plants in the control group. After some time, the plants'
heights (in cm) are measured, resulting in the following data:

\begin{longtable}[]{@{}ccccccc@{}}
\toprule
& value1 & value2 & value3 & value4 & value5 & value6\tabularnewline
\midrule
\endhead
sample 1: & 16.4 & 14.2 & 19.4 & 17.3\tabularnewline
sample 2: & 10.3 & 9.9 & 9.4 & 11 & 10.4 & 10.7\tabularnewline
\bottomrule
\end{longtable}
\begin{answerlist}
  \item Determine degrees of freedom.
  \item Determine \(t^\star\) for a \(98\%\) confidence interval.
  \item Determine \(SE\).
  \item Determine a lower bound of the \(98\%\) confidence interval of
\(\mu_2-\mu_1\).
  \item Determine an upper bound of the \(98\%\) confidence interval of
\(\mu_2-\mu_1\).
  \item Determine \(|t_\text{obs}|\) under the null hypothesis
\(\mu_2-\mu_1=0\).
  \item Determine a lower bound of the two-tail \(p\)-value.
  \item Determine an upper bound of two-tail \(p\)-value.
  \item Do you reject the null hypothesis with a two-tail test using a
significance level \(\alpha = 0.02\)? (yes or no)
\end{answerlist}
\end{question}

\begin{solution}
These data are unpaired. We might as well find the sample means and
sample standard deviations (use a calculator's built-in function for
standard deviation). \[\overline{x_1} = 16.8 \]
\[\overline{x_2} = 10.3 \] \[s_1 = 2.15 \] \[s_2 = 0.571 \]

We make a conservative estimate of the degrees of freedom using the
appropriate formula. \[df ~=~ \min(n_1,\,n_2)-1 ~=~ \min(4,6)-1 ~=~ 3 \]
We use the \(t\) table to find \(t^\star\) such that
\(P(|T|<t^\star) = 0.98\) \[t^\star = 4.54 \] We use the \(SE\) formula
for unpaired data.
\[SE = \sqrt{\frac{(s_1)^2}{n_1}+\frac{(s_2)^2}{n_2}} =
\sqrt{\frac{(2.15)^2}{4}+\frac{(0.571)^2}{6}} = 1.1 \] We find the
bounds of the confidence interval.
\[CI ~=~ (\overline{x_2}-\overline{x_1})\pm t^{\star} SE\]
\[CI ~=~ (-11.494,\, -1.506) \] We find \(t_\text{obs}\).
\[t_\text{obs} = \frac{(\overline{x_2}-\overline{x_1})-(\mu_2-\mu_1)_0}{SE} = \frac{(10.3-16.8)-0}{1.1} = -5.91\]
We find \(|t_\text{obs}|\). \[|t_\text{obs}| = 5.91 \] We use the table
to determine bounds on \(p\)-value. Remember, \(df=3\) and
\(p\text{-value} = P(|T|>|t_\text{obs}|)\).
\[0.005 ~<~ p\text{-value} ~<~ 0.01\] We should consider both
comparisons to make our decision. \[|t_\text{obs}| > t^{\star} \]
\[p\text{-value} < \alpha \] Thus, we reject the null hypothesis. Also
notice the confidence interval does not contain 0.
\begin{answerlist}
  \item 3
  \item 4.54
  \item 1.1
  \item -11.494
  \item -1.506
  \item 5.909
  \item 0.005
  \item 0.01
  \item yes
\end{answerlist}
\end{solution}

