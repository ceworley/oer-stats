\documentclass[]{article}
\usepackage{lmodern}
\usepackage{amssymb,amsmath}
\usepackage{ifxetex,ifluatex}
\usepackage{fixltx2e} % provides \textsubscript
\ifnum 0\ifxetex 1\fi\ifluatex 1\fi=0 % if pdftex
  \usepackage[T1]{fontenc}
  \usepackage[utf8]{inputenc}
\else % if luatex or xelatex
  \ifxetex
    \usepackage{mathspec}
  \else
    \usepackage{fontspec}
  \fi
  \defaultfontfeatures{Ligatures=TeX,Scale=MatchLowercase}
\fi
% use upquote if available, for straight quotes in verbatim environments
\IfFileExists{upquote.sty}{\usepackage{upquote}}{}
% use microtype if available
\IfFileExists{microtype.sty}{%
\usepackage{microtype}
\UseMicrotypeSet[protrusion]{basicmath} % disable protrusion for tt fonts
}{}
\usepackage[margin=1in]{geometry}
\usepackage{hyperref}
\hypersetup{unicode=true,
            pdfborder={0 0 0},
            breaklinks=true}
\urlstyle{same}  % don't use monospace font for urls
\usepackage{graphicx,grffile}
\makeatletter
\def\maxwidth{\ifdim\Gin@nat@width>\linewidth\linewidth\else\Gin@nat@width\fi}
\def\maxheight{\ifdim\Gin@nat@height>\textheight\textheight\else\Gin@nat@height\fi}
\makeatother
% Scale images if necessary, so that they will not overflow the page
% margins by default, and it is still possible to overwrite the defaults
% using explicit options in \includegraphics[width, height, ...]{}
\setkeys{Gin}{width=\maxwidth,height=\maxheight,keepaspectratio}
\IfFileExists{parskip.sty}{%
\usepackage{parskip}
}{% else
\setlength{\parindent}{0pt}
\setlength{\parskip}{6pt plus 2pt minus 1pt}
}
\setlength{\emergencystretch}{3em}  % prevent overfull lines
\providecommand{\tightlist}{%
  \setlength{\itemsep}{0pt}\setlength{\parskip}{0pt}}
\setcounter{secnumdepth}{0}
% Redefines (sub)paragraphs to behave more like sections
\ifx\paragraph\undefined\else
\let\oldparagraph\paragraph
\renewcommand{\paragraph}[1]{\oldparagraph{#1}\mbox{}}
\fi
\ifx\subparagraph\undefined\else
\let\oldsubparagraph\subparagraph
\renewcommand{\subparagraph}[1]{\oldsubparagraph{#1}\mbox{}}
\fi

%%% Use protect on footnotes to avoid problems with footnotes in titles
\let\rmarkdownfootnote\footnote%
\def\footnote{\protect\rmarkdownfootnote}

%%% Change title format to be more compact
\usepackage{titling}

% Create subtitle command for use in maketitle
\providecommand{\subtitle}[1]{
  \posttitle{
    \begin{center}\large#1\end{center}
    }
}

\setlength{\droptitle}{-2em}

  \title{}
    \pretitle{\vspace{\droptitle}}
  \posttitle{}
    \author{}
    \preauthor{}\postauthor{}
    \date{}
    \predate{}\postdate{}
  

\begin{document}

Let's remind ourselves where the formulas for standard error arise. If
we have a population described by random variable \(X\), then that
population's mean is \(E(X)\) and that population's standard deviation
is \(\sqrt{\text{Var}(X)\).

\subsection{Problem 1}\label{problem-1}

\begin{enumerate}
\def\labelenumi{\arabic{enumi}.}
\tightlist
\item
  There is a normal population \(X\).
  \[X ~\sim~ \mathcal{N}\left(\mu=20\,,~\sigma=5 \right) \] Another
  population \(Y\) is determined by \(X\).
  \[Y ~\sim~ \frac{X}{7}+\frac{X}{7}+\frac{X}{7}+\frac{X}{7}+\frac{X}{7}+\frac{X}{7}+\frac{X}{7} \]
\end{enumerate}

\subsection{Problem 2}\label{problem-2}

\begin{enumerate}
\def\labelenumi{\arabic{enumi}.}
\setcounter{enumi}{1}
\tightlist
\item
  You have two populations (random variables): \(V\) and \(W\).
  \[V ~~\sim~~ \mathcal{N}\left(\mu=99\,,~\sigma=31 \right)\]
  \[W ~~\sim~~ \mathcal{N}\left(\mu=77\,,~\sigma=11 \right)\] A (normal)
  population \(X\) is determined by \(V\) and \(W\).
  \[X ~~\sim~~ \left(\frac{V}{3}+\frac{V}{3}+\frac{V}{3}\right)-\left(\frac{W}{6}+\frac{W}{6}+\frac{W}{6}+\frac{W}{6}+\frac{W}{6}+\frac{W}{6}\right)\]

  \begin{enumerate}
  \def\labelenumii{\alph{enumii}.}
  \tightlist
  \item
    Evaluate \(\text{E}(X)\).
  \item
    Evaluate \(\text{Var}(X)\).
  \item
    Evaluate \(P(X>25)\).
  \item
    Determine \(x\) such that \(P(X<x) = 0.888\).
  \end{enumerate}
\end{enumerate}

\newpage

\subsection{Solution 2}\label{solution-2}

\begin{enumerate}
\def\labelenumi{\arabic{enumi}.}
\setcounter{enumi}{1}
\item
  \begin{enumerate}
  \def\labelenumii{\alph{enumii}.}
  \item
    Expected value follows basic rules. \(E(aA+bB) = aE(A)+bE(B)\)
    \[E(X) ~=~ 3 \left(\frac{E(V)}{3}\right) - 6 \left(\frac{E(W)}{6}\right) ~=~ E(V)-E(W) ~=~ 99-77 ~=~22\]
  \item
    Variance has a more complicated rule.
    \(Var(aA+bB)=a^2Var(A)+b^2Var(B)\)

    \begin{align*}
    Var(X) ~&=~ 3 \left(\frac{Var(V)}{9}\right) + 6 \left(\frac{Var(W)}{36}\right)\\
    &=~ \frac{Var(V)}{3}+\frac{Var(W)}{6} \\
    &=~ \frac{31^2}{3}+\frac{11^2}{6} \\
    &=~340.5
    \end{align*}
  \item
  \end{enumerate}
\end{enumerate}

\newpage

\subsection{Problem 3}\label{problem-3}

\begin{enumerate}
\def\labelenumi{\arabic{enumi}.}
\setcounter{enumi}{2}
\tightlist
\item
  You have two populations (random variables): \(V\) and \(W\).
  \[V ~~\sim~~ \mathcal{N}\left(\mu=99\,,~\sigma=31 \right)\]
  \[W ~~\sim~~ \mathcal{N}\left(\mu=77\,,~\sigma=11 \right)\] A (normal)
  population \(Y\) is determined by \(V\) and \(W\).
  \[Y ~~\sim~~ \frac{(V-W)+(V-W)+(V-W)+(V-W)}{4}\]

  \begin{enumerate}
  \def\labelenumii{\alph{enumii}.}
  \tightlist
  \item
    Evaluate \(\text{E}(Y)\).
  \item
    Evaluate \(\text{Var}(Y)\).
  \item
    Evaluate \(P(Y>25)\).
  \item
    Determine \(y\) such that \(P(Y<y) = 0.888\).
  \end{enumerate}
\end{enumerate}


\end{document}
