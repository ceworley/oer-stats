
\begin{question}
An experiment is run with a treatment group of size 157 and a control
group of size 180. The results are summarized in the table below.

\begin{longtable}[]{@{}ccc@{}}
\toprule
& treatment & control\tabularnewline
\midrule
\endhead
special & 52 & 77\tabularnewline
not special & 105 & 103\tabularnewline
\bottomrule
\end{longtable}

Using a significance level of 0.04, determine whether the treatment
causes an effect on the proportion of cases that are special.
\begin{answerlist}
  \item Determine a \(p\)-value.
  \item Does the treatment have a significant effect? (yes or no)
\end{answerlist}
\end{question}

\begin{solution}
State the hypotheses. \[H_0:~~ p_2-p_1 = 0 \]
\[H_\text{A}: ~~ p_2-p_1 \ne 0 \] Determine the sample proportions.
\[\hat{p}_1 = \frac{52}{157} = 0.331 \]
\[\hat{p}_2 = \frac{77}{180} = 0.428 \] Determine the difference of
sample proportions. \[\hat{p}_2-\hat{p}_1 = 0.428-0.331 = 0.097 \]
Determine the pooled proportion (because the null assumes the population
proportions are equal). \[ \hat{p} = \frac{52+77}{157+180} = 0.383 \]
Determine the standard error. \[
\begin{aligned}
SE &= \sqrt{\frac{\hat{p}(1-\hat{p})}{n_1}+\frac{\hat{p}(1-\hat{p})}{n_2}} \\[1em]
&= \sqrt{\frac{(0.383)(0.617)}{157}+\frac{(0.383)(0.617)}{180}} \\[1em]
&= 0.0531
\end{aligned}
\] We can be more specific about what the null hypothesis claims.
\[H_0:~~~~ \hat{P}_2-\hat{P}_1 ~\sim~ \mathcal{N}(0,\,0.0531)\] We want
to describe how unusual our observation is under the null by finding the
\(p\)-value. To do so, first find the \(z\) score. \[
\begin{aligned}
z &= \frac{(\hat{p}_2-\hat{p}_1)-(p_2-p_1)_0}{SE}\\
&= \frac{(0.428-0.331)-0}{0.0531} \\
&= 1.83
\end{aligned}
\] Determine the \(p\)-value. \[
\begin{aligned}
p\text{-value} &= 2\cdot \Phi(-|z|)\\
&= 2\cdot \Phi(-1.83)\\
&= 0.0672
\end{aligned}
\] Compare the \(p\)-value to the signficance level.
\[ p\text{-value} > \alpha \] So, we retain the null hypothesis. Thus
the difference in proportions is not significant.
\begin{answerlist}
  \item The \(p\)-value = 0.0672
  \item We retain the null, so no
\end{answerlist}
\end{solution}

