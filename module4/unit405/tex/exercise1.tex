
\begin{question}
An experiment is run with a treatment group of size 221 and a control
group of size 270. The results are summarized in the table below.

\begin{longtable}[c]{@{}ccc@{}}
\toprule
& treatment & control\tabularnewline
\midrule
\endhead
cold & 112 & 111\tabularnewline
not cold & 109 & 159\tabularnewline
\bottomrule
\end{longtable}

Using a significance level of 0.04, determine whether the treatment
causes an effect on the proportion of cases that are cold.
\begin{answerlist}
  \item Determine a \(p\)-value.
  \item Does the treatment have a significant effect? (yes or no)
\end{answerlist}
\end{question}

\begin{solution}
State the hypotheses. \[H_0:~~ p_2-p_1 = 0 \]
\[H_\text{A}: ~~ p_2-p_1 \ne 0 \] Determine the sample proportions.
\[\hat{p}_1 = \frac{112}{221} = 0.507 \]
\[\hat{p}_2 = \frac{111}{270} = 0.411 \] Determine the difference of
sample proportions. \[\hat{p}_2-\hat{p}_1 = 0.411-0.507 = -0.096 \]
Determine the pooled proportion (because the null assumes the population
proportions are equal). \[ \hat{p} = \frac{112+111}{221+270} = 0.454 \]
Determine the standard error. \[
\begin{aligned}
SE &= \sqrt{\frac{\hat{p}(1-\hat{p})}{n_1}+\frac{\hat{p}(1-\hat{p})}{n_2}} \\[1em]
&= \sqrt{\frac{(0.454)(0.546)}{221}+\frac{(0.454)(0.546)}{270}} \\[1em]
&= 0.0452
\end{aligned}
\] We can be more specific about what the null hypothesis claims.
\[H_0:~~~~ \hat{P}_2-\hat{P}_1 ~\sim~ \mathcal{N}(0,\,0.0452)\] We want
to describe how unusual our observation is under the null by finding the
\(p\)-value. To do so, first find the \(z\) score. \[
\begin{aligned}
z &= \frac{(\hat{p}_2-\hat{p}_1)-(p_2-p_1)_0}{SE}\\
&= \frac{(0.411-0.507)-0}{0.0452} \\
&= -2.12
\end{aligned}
\] Determine the \(p\)-value. \[
\begin{aligned}
p\text{-value} &= 2\cdot \Phi(-|z|)\\
&= 2\cdot \Phi(-2.12)\\
&= 0.034
\end{aligned}
\] Compare the \(p\)-value to the signficance level.
\[ p\text{-value} < \alpha \] So, we reject the null hypothesis. Thus
the difference in proportions is significant.
\begin{answerlist}
  \item The \(p\)-value = 0.034
  \item We reject the null, so yes
\end{answerlist}
\end{solution}

