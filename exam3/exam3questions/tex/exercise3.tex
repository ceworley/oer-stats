
\begin{question}
You are interested in whether a treatment causes an effect on a
continuously measurable attribute. You use a treatment group with 5
cases and a control group with 5 cases. You decide to run a hypothesis
test with a significance level of 0.05. Your data is below. Please use 5
for the degrees of freedom (calculated with the Welch-Satterthwaite
equation).

\begin{longtable}[]{@{}cc@{}}
\toprule
treatment & control\tabularnewline
\midrule
\endhead
440 & 650\tabularnewline
420 & 400\tabularnewline
460 & 510\tabularnewline
450 & 580\tabularnewline
460 & 690\tabularnewline
\bottomrule
\end{longtable}
\begin{answerlist}
  \item State the null hypothesis.
  \item State the alternative hypothesis.
  \item Evaluate the critical value. (The critical value is either \(z^\star\)
or \(t^\star\). Determine its value.)
  \item Determine the standard error of the relevant sampling distribution.
  \item Evaluate the absolute value of the test statistic. (The test statistic
is either \(z_\text{obs}\) or \(t_\text{obs}\). Determine its absolute
value.)
  \item If possible, evaluate the \(p\)-value. Otherwise, describe an interval
containing the \(p\)-value.
  \item Do we reject or retain the null?
\end{answerlist}
\end{question}

\begin{solution}
We are given unpaired data. We are considering a difference of means.
Label the given information. \[
\begin{aligned}
n_1 &= 5 \\
n_2 &= 5 \\
\alpha &= 0.05
\end{aligned}
\] State the hypotheses. \[H_0:~~\mu_2-\mu_1 = 0\]
\[H_A:~~\mu_2-\mu_1 \ne 0\] We are using a two-tail test. Find
\(t^{\star}\) such that \(P(|T|>t^{\star})=0.05\) by using a \(t\)
table. \[t^{\star} = 2.57\] Determine the sample statistics. Use a
calculator! \[
\begin{aligned}
\bar{x}_1 &= 446 \\
s_1 &= 16.7 \\
\bar{x}_2 &= 566 \\
s_2 &= 115
\end{aligned}
\] Calculate the standard error. \[
\begin{aligned}
SE &= \sqrt{\frac{(s_1)^2}{n_1}+\frac{(s_2)^2}{n_2}} \\
&= \sqrt{\frac{(16.7)^2}{5}+\frac{(115)^2}{5}} \\
&= 52
\end{aligned}
\] Determine the test statistic. \[
\begin{aligned}
t_\text{obs} &= \frac{(\bar{x}_2-\bar{x}_1)-(\mu_2-\mu_1)_0}{SE} \\
&= \frac{(566-446)-(0)}{52} \\
&= 2.31
\end{aligned}
\] Compare \(|t_\text{obs}|\) and \(t^\star\).
\[|t_\text{obs}| < t^\star \] We can determine an interval for the
\(p\)-value using the \(t\) table. \[0.05 < p\text{-value} < 0.1\]
Compare \(p\)-value and \(\alpha\). \[p\text{-value} > \alpha \] We
conclude that we should retain the null hypothesis.
\begin{answerlist}
  \item \(H_0: ~~\mu_2-\mu_1=0\)
  \item \(H_\text{A}:~~ \mu_2-\mu_1\ne 0\)
  \item \(t^\star = 2.57\)
  \item \(SE = 52\)
  \item \(|t_\text{obs}| = 2.31\)
  \item \(0.05 < p\text{-value} < 0.1\)
  \item retain the null
\end{answerlist}
\end{solution}

