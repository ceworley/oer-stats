
\begin{question}
An experiment is run with a treatment group of size 196 and a control
group of size 208. The results are summarized in the table below.

\begin{longtable}[]{@{}ccc@{}}
\toprule
& treatment & control\tabularnewline
\midrule
\endhead
reclusive & 23 & 38\tabularnewline
not reclusive & 173 & 170\tabularnewline
\bottomrule
\end{longtable}

Using a significance level of 0.1, determine whether the treatment
causes an effect on the proportion of cases that are reclusive.
\begin{answerlist}
  \item State the null hypothesis.
  \item State the alternative hypothesis.
  \item Evaluate the critical value. (The critical value is either \(z^\star\)
or \(t^\star\). Determine its value.)
  \item Determine the standard error of the relevant sampling distribution.
  \item Evaluate the absolute value of the test statistic. (The test statistic
is either \(z_\text{obs}\) or \(t_\text{obs}\). Determine its absolute
value.)
  \item If possible, evaluate the \(p\)-value. Otherwise, describe an interval
containing the \(p\)-value.
  \item Do we reject or retain the null?
\end{answerlist}
\end{question}

\begin{solution}
State the hypotheses. \[H_0:~~ p_2-p_1 = 0 \]
\[H_\text{A}: ~~ p_2-p_1 \ne 0 \] Find \(z^\star\) such that
\(P(|Z|>z^\star) = 0.1\).
\[z^\star = \Phi^{-1}\left(1-\frac{\alpha}{2}\right) = 1.64 \] Determine
the sample proportions. \[\hat{p}_1 = \frac{23}{196} = 0.117 \]
\[\hat{p}_2 = \frac{38}{208} = 0.183 \] Determine the difference of
sample proportions. \[\hat{p}_2-\hat{p}_1 = 0.183-0.117 = 0.066 \]
Determine the pooled proportion (because the null assumes the population
proportions are equal). \[ \hat{p} = \frac{23+38}{196+208} = 0.151 \]
Determine the standard error. \[
\begin{aligned}
SE &= \sqrt{\frac{\hat{p}(1-\hat{p})}{n_1}+\frac{\hat{p}(1-\hat{p})}{n_2}} \\[1em]
&= \sqrt{\frac{(0.151)(0.849)}{196}+\frac{(0.151)(0.849)}{208}} \\[1em]
&= 0.0356
\end{aligned}
\] We can be more specific about what the null hypothesis claims.
\[H_0:~~~~ \hat{P}_2-\hat{P}_1 ~\sim~ \mathcal{N}(0,\,0.0356)\] We want
to describe how unusual our observation is under the null by finding the
\(p\)-value. To do so, first find the \(z\) score. \[
\begin{aligned}
z &= \frac{(\hat{p}_2-\hat{p}_1)-(p_2-p_1)_0}{SE}\\
&= \frac{(0.183-0.117)-0}{0.0356} \\
&= 1.85
\end{aligned}
\] Determine the \(p\)-value. \[
\begin{aligned}
p\text{-value} &= 2\cdot \Phi(-|z|)\\
&= 2\cdot \Phi(-1.85)\\
&= 0.0644
\end{aligned}
\] Compare the \(p\)-value to the signficance level.
\[ p\text{-value} < \alpha \] So, we reject the null hypothesis. Thus
the difference in proportions is significant.
\begin{answerlist}
  \item \(H_0: ~~p_2-p_1=0\)
  \item \(H_\text{A}:~~ p_2-p_1\ne 0\)
  \item \(z^\star = 1.64\)
  \item \(SE = 0.0356\)
  \item \(|z_\text{obs}| = 1.85\)
  \item \(p\text{-value} = 0.0644\)
  \item reject the null
\end{answerlist}
\end{solution}

