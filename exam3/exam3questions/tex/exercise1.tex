
\begin{question}
As an ornithologist, you wish to determine the average body mass of
\emph{Dendroica castanea}. You randomly capture 20 adults of
\emph{Dendroica castanea}, resulting in a sample mean of 15.13 grams and
a sample standard deviation of 1.16 grams. You decide to report a 80\%
confidence interval.
\begin{answerlist}
  \item Determine the lower bound of the confidence interval.
  \item Determine the upper bound of the confidence interval.
\end{answerlist}
\end{question}

\begin{solution}
We are given the sample size, sample mean, sample standard deviation,
and confidence level. \[
\begin{aligned}
n &= 20 \\
\bar{x} &= 15.13 \\
s &= 1.16\\
CL &= 0.8
\end{aligned}
\] Determine the degrees of freedom (because we don't know \(\sigma\)
and we are doing inference so we need to use the \(t\) distribution).
\[df = n-1 = 19\] Determine the critical \(t\) value, \(t^\star\), such
that \(P(|T|<t^\star) = 0.8\). \[t^\star = 1.33 \] Calculate the
standard error.
\[ SE = \frac{s}{\sqrt{n}} = \frac{1.16}{\sqrt{20}} = 0.259 \] We want
to make an inference about the population mean.
\[\mu ~\approx~ \bar{x}\pm t^\star SE \] Determine the bounds. \[
\begin{aligned}
CI &= (\bar{x} - t^\star SE,\,\bar{x} + t^\star SE) \\
   &= (15.13 - 1.33\times0.259,\,15.13 + 1.33\times0.259) \\
   &= (14.8,\,15.5) 
\end{aligned}
\] We are 80\% confident that the population mean is between 14.8 and
15.5.
\begin{answerlist}
  \item Lower bound = 14.8
  \item Upper bound = 15.5
\end{answerlist}
\end{solution}

