
\begin{question}
A teacher has 7 students who have each taken two quizzes. Perform a
two-tail test with significance level 0.1 to determine whether students'
performance changed on average.

\begin{longtable}[]{@{}cccccccc@{}}
\toprule
& student1 & student2 & student3 & student4 & student5 & student6 &
student7\tabularnewline
\midrule
\endhead
quiz 1: & 70.6 & 53.6 & 55.3 & 86.2 & 88.4 & 75 & 77.4\tabularnewline
quiz 2: & 67.2 & 46.1 & 54.3 & 83 & 83.7 & 70.8 & 81.4\tabularnewline
\bottomrule
\end{longtable}
\begin{answerlist}
  \item State the null hypothesis.
  \item State the alternative hypothesis.
  \item Evaluate the critical value. (The critical value is either \(z^\star\)
or \(t^\star\). Determine its value.)
  \item Determine the standard error of the relevant sampling distribution.
  \item Evaluate the absolute value of the test statistic. (The test statistic
is either \(z_\text{obs}\) or \(t_\text{obs}\). Determine its absolute
value.)
  \item If possible, evaluate the \(p\)-value. Otherwise, describe an interval
containing the \(p\)-value.
  \item Do we reject or retain the null?
\end{answerlist}
\end{question}

\begin{solution}
We are given paired data. We are considering a mean of differences.
Label the given information. \[
\begin{aligned}
n &= 7 \\
\alpha &= 0.1
\end{aligned}
\] State the hypotheses. \[H_0: ~~\mu_\text{diff}=0\]
\[H_\text{A}: ~~\mu_\text{diff}\ne0\] Determine the degrees of freedom.
\[df = n-1 = 6 \] We determine \(t^\star\) such that
\(P(|T|>t^\star) = 0.1\). \[t^\star = 1.94 \] Subtract each student's
scores to get the differences.

\begin{longtable}[]{@{}cccccccc@{}}
\toprule
& student1 & student2 & student3 & student4 & student5 & student6 &
student7\tabularnewline
\midrule
\endhead
quiz2-quiz1: & -3.4 & -7.5 & -1 & -3.2 & -4.7 & -4.2 & 4\tabularnewline
\bottomrule
\end{longtable}

Find the sample mean. \[\overline{x_\text{diff}} = -2.86\] Find the
sample standard deviation. \[s_\text{diff} = 3.6\] Determine the
standard error. \[SE = \frac{s_\text{diff}}{\sqrt{n}} = 1.36 \]
Calculate the observed \(t\) score.
\[t_\text{obs} = \frac{\overline{x_\text{diff}}-(\mu_\text{diff})_0}{SE} = \frac{-2.86-0}{1.36} = -2.103 \]
Compare \(|t_\text{obs}|\) and \(t^\star\).
\[|t_\text{obs}| > t^\star \] We can determine an interval for the
\(p\)-value using the \(t\) table. \[0.05 < p\text{-value} < 0.1\] We
conclude that we should reject the null hypothesis.
\begin{answerlist}
  \item \(H_0: ~~\mu_\text{diff}=0\)
  \item \(H_\text{A}:~~ \mu_\text{diff} \ne 0\)
  \item \(t^\star = 1.94\)
  \item \(SE = 1.36\)
  \item \(|t_\text{obs}| = 2.103\)
  \item \(0.05 < p\text{-value} < 0.1\)
  \item reject the null
\end{answerlist}
\end{solution}

